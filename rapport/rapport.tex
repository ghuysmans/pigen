\documentclass[10pt,a4paper]{article}
\usepackage[utf8]{inputenc}
\usepackage[francais]{babel}
\usepackage[T1]{fontenc}
\usepackage{amsmath}
\usepackage{amsfonts}
\usepackage{amssymb}
\usepackage{graphicx}
\usepackage[margin=1.5cm]{geometry}
\author{\textsc{Huysmans} Guillaume, \textsc{Lempereur} Martin}
\title{Le rapport de la simulation simulée}
\begin{document}
\maketitle

\section{Introduction}
Le CPU d'un ordinateur est entièrement déterministe. Certains algorithmes
utilisés par exemple en intelligence artificielle ne le sont pas et auront de
bien meilleurs résultats lorsqu'il pourront explorer <<~au hasard~>> différentes
zones de leur espace de recherche. Différentes méthodes existent afin de générer
des nombres pseudo-aléatoires (ils ne le sont pas vraiment puisqu'ils sont
implémentés avec des algorithmes parfaitement déterministes). La seule donnée
peu prévisible que nous choisissons d'utiliser est le moment auquel le générateur
est lancé.
Les méthodes vues au cours afin de générer des nombres selon des lois normales,
poissonniennes, exponentielles, etc. se basent sur un générateur uniforme et
c'est ce que nous essaierons de construire ici. 

\section{Tests}
L'erreur de première espèce (risque de confirmer l'hypothèse de même loi à tort)
est fixée à $\alpha=0.05$.

\subsection{Histogramme}
Un histogramme peut nous donner un premier aperçu sur la répartition
du premier million de décimales du nombre $\pi$.
Cependant un histogramme n'a aucune valeur scientifique \textit{per se}. 
On peut facilement comprendre que la séquence [0,1,2,3,4,5,6,7,8,9]
nous donnera un histogramme parfaitement régulier.
Nous avons donc implémenté une batterie de tests qui se concentrent sur d'autres aspects de l'uniformité.

%FIXME résultats

\subsection{Test du Poker}
Nous avons choisi de nous baser directement sur les décimales contenues
dans le fichier. Ainsi, nous utiliserons 10 classes qui correspondront
à chaque fois au nombre de <<~faces~>> différentes obtenues
après $k$ lancers afin d'éviter de trop importantes erreurs d'arrondi.
%FIXME pluriel arrondi ?
Ce test est répété $n$ fois et un ratio de tests réussis est calculé.
Des classes peu peuplées ont été regroupées afin d'éviter
des valeurs attendues nulles.

\texttt{./poker.py 10 1000 <pi6.txt} fonctionne à 91\%.

\subsection{Test de Gap}
%FIXME autres paramètres
On décide d'un intervalle $[a,b] \subset [0,1]$,
on génère une séquence de nombres (nous avons décidé de regrouper des décimales)
et on marque ceux compris dans cet intervalle. Le but du test est de regarder les
distances entre deux nombres marqués successifs.
Afin d'éviter des classes aux valeurs attendues nulles
qui auraient une influence énorme sur le test du $\chi^2$,
les paramètres ont été soigneusement choisis.

\texttt{./gap.py 11 10000 20 <pi6.txt} fonctionne à 85\%.

\subsection{Test du collectionneur de coupons}
On essaie de collectionner $s$ coupons (par défaut 10 -- un par chiffre)
jusqu'à maximum $m$ tirages.
On crée une classe par nombre de tirages nécessaires pour les collectionner tous.
Il n'y en aura pas pour moins de $s$ lancers puisqu'on veut obtenir
$s$ coupons différents.
Une dernière classe regroupe les longueurs strictement supérieures à $m$.
Le test est répété $t$ fois sur des séquences consécutives dans le fichier.

\texttt{./coupon.py 1000 15 1000 <pi6.txt} fonctionne à 95\%.

\section{Générateur}
Puisque les décimales de $\pi$ semblent proches d'une séquence de chiffres
pseudo-aléatoire, nous pouvons les utiliser afin d'obtenir un \textit{seed}
pour un LCG\footnote{Linear Congruential Generator}. Ce dernier ne sera utilisé
que $k=100$ fois avant d'utiliser la séquence suivante. Celle-ci est composée de
$d=10$ chiffres qui forment ensemble un (grand) nombre entier.
Le multiplicateur du générateur est un nombre premier de 10 chiffres testé
à l'aide de l'outil \texttt{factor} issu des \textit{GNU coreutils}.
L'incrément a été fixé à 42 de façon totalement arbitraire.
Le module vaut $2^{20}$ ce qui permet de calculer rapidement par masquage
le reste de la division.
Les chiffres générés sont uniquement le premier dans les décimales de chaque
valeur de départ divisée par $m$.
Puisque le numérateur est par définition toujours strictement inférieur à $m$,
c'est un rationnel dans $]0,1[$ qui est utilisé.

La position de départ dans le fichier contenant le million de décimales
est déterminée par la date en secondes. Lorsqu'on arrive au bout de ce fichier,
on recommence au début.

Les résultats des tests sur les chiffres générés avec les mêmes paramètres
qu'à la section précédente se trouvent à la figure \ref{fig:genD}.

\begin{figure}
\center
\begin{tabular}{l|c|c|c}
Test & $\pi$ & Générés & Résultat \\ \hline \hline
Histogramme & FIXME & FIXME & ?? \\ \hline
Poker & 91\% & 98\% & > \\ \hline
Gap & 85\% & 85\% & = \\ \hline
Coupons & 95\% & 94\% & <
\end{tabular}
\caption{Comparaison des chiffres générés et des premières décimales de $\pi$}
\label{fig:genD}
\end{figure}

ON VEUT DES CHIFFRES. PLEIN.

\end{document}